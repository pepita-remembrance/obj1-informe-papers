\documentclass[a4paper,10pt]{article}

\usepackage[utf8]{inputenc}
\usepackage[T1]{fontenc}
\usepackage[spanish]{babel}
\usepackage{fullpage}
\usepackage{csquotes}
\usepackage[backend=biber,style=numeric,sorting=ynt]{biblatex}
\usepackage{xcolor}
\usepackage{hyperref}

\newcommand{\borrador}[1]{{\color{red}{#1}}}
\newcommand\nombre{\textsc}

\addbibresource{main.bib}

\setlength\parindent{1em}

\title{Comentarios sobre modelado con clases y prototipos en el paradigma orientado a objetos}
\author{Federico Aloi y Ariel Álvarez}

\begin{document}

\maketitle

\section{Introducción}
Una de las principales discusiones\cite{Treaty} que se enmarcan dentro del universo de la programación orientada a objetos radica en cómo modelar el comportamiento (es decir, las acciones que pueden realizar los objetos). A lo largo de la corta historia del paradigma, han surgido distintas alternativas al problema, entre las que destacaremos dos: la utilización de una jerarquía de clases y el modelado con prototipos.

A partir de la lectura y el análisis de trabajos fundacionales sobre ambas opciones, es nuestro objetivo poder resumir las ideas que allí se mencionan, para luego elaborar nuestras propias conclusiones y plasmar también nuestra opinión sobre el tema.

\section{Resumen de los trabajos leidos}

\section{Conclusiones}

\section{Opiniones}

\printbibliography

\end{document}
